\subsection{Reinterpreting identity as indiscernibility}

We start by assuming that we are given an identity relation $\approx$,
  which partitions the subject terms $S_G$ according to
  \mbox{equation} \ref{eq:equivalence_set}.

\begin{equation}
\label{eq:equivalence_set}
  \equivset{x}
=
  \setdef{
    y \in S_G
  }{
    \equivpair{x}{y}
  }
\end{equation}

\noindent Identity can be defined as the smallest equivalence relation,
  i.e. the most fine-grained partition of $S_G$.
For reasoning purposes, the fact that $\approx$ is an equivalence relation
  is important, allowing symmetrical and transitive inferences.
Identity implies indiscernibility with respect to all properties.

We can generalize the notion of indiscernibility
  by parameterizing the set of properties with respect to which
  indiscernibility is determined.
According to this generalization,
  resources $x$ and $y$ are indiscernible with respect to
  a set of properties $PO \subseteq P_G \times O_G$
  iff $\forall po \in PO (po(x) \leftrightarrow po(y))$ is the case.
Every indiscernibility relation is also an equivalence relation,
  although not necessarily the smallest one.
Moreover, every indiscernibility relation defined over the domain $S_G$
  is also an identity relation, but over a different domain.

We now reinterpret the identity relation $\approx$
  as if it were an indiscernibility relation
  whose set of properties $PO$ is implicit.
Based on the extensional specification of the identity relation,
  we make the set of properties with respect to which it is indiscernible
  explicit.
Definition \ref{def:indiscernibility_properties} makes explicit
  the properties relative to which the terms $x_i$ are indiscernibile.

\begin{definition}[Indiscernibility properties]
\label{def:indiscernibility_properties}
\begin{align}
  \indpo_{\approx}(\set{\range{x_1}{x_n}})
=
  \setdef{
    \pair{p}{o} \in P_G \times O_G
  }{\nonumber\\
    \bigwedge_{1 \leq i \leq n}
      \exists p_i \in \equivset{p},
        \exists o_i \in \equivset{o}(
          \triple{x_i}{p_i}{o_i} \in G
        )
  }\nonumber
\end{align}
\end{definition}

\noindent Notice that in definition \ref{def:indiscernibility_properties}
  we close both the predicate terms $p$ and the object terms $o$
  under identity.
Performing these closures is important in order to identify
  the relevant indiscernibility properties.

In the above, we were interested in the properties
  that resources share with one other.
But we are also interested in the predicates that are shared by
  a set of resources.
This amounts to a simple abstraction of
  definition \ref{def:indiscernibility_properties},
  equating the sets of objects (closed under identity)
  and only returning the set of shared RDF predicate terms
  (see definition \ref{def:indiscernibility_predicates}).

\small
\begin{definition}[Indiscernibility predicates]
\label{def:indiscernibility_predicates}
\begin{align}
  \indp_{\approx}(\setrange{x_1}{x_n})
=
  \setdef{
    p \in P_G
  }{
    \exists_{\range{p_1}{p_n} \in \equivset{p}}(\nonumber\\
        \equivset{
          \setdef{
            o \in O_G
          }{
            \triple{x_1}{p_1}{o}
          }
        }
      =
        \ldots
      =
        \equivset{
          \setdef{
            o \in O_G
          }{
            \triple{x_n}{p_n}{o}
          }
        }
    )
  }\nonumber
\end{align}
\end{definition}
\normalsize



\subsection{Discerning the same}

In the previous section we saw that resources are \mbox{\emph{indiscernible}}
  with respect to $PO$ iff they cannot be told apart
  in a language that only contains the properties denoted by $PO$
  (the so-called indiscernibility properties):

In the same vein,
  and builing upon definition \ref{def:indiscernibility_predicates},
  we say that two pairs of resources are \emph{simi-discernible}
  iff their \mbox{indiscernibility} predicates $P \subseteq P_G$ are the same.

When we look at the pairs that constitute (the extension of)
  an identity relation, all identity assertions look the same.
But when we take the considerations of the previous section into account,
  we see that within a given identity relation
  there are pairs that assert indiscernibility
  based on different domain predicates.
Stating this formally,
  simi-discernibility is an equivalence relation on pairs of resources,
  which induces a partition of the Cartesian product of the domain.
Definition \ref{def:simidiscernibility_relation} makes this concrete
  in terms of the earlier definitions.

\begin{definition}[Simi-discernibility relation]
\label{def:simidiscernibility_relation}
\begin{align}
  \equiv_{\indp_{\approx}}
=
  \setdef{
    \pair{\pair{x_1}{x_2}}{\pair{y_1}{y_2}} \in (S_G^2)^2
  }{\nonumber\\
    \indp_{\approx}(\set{x_1,x_2}) = \indp_{\approx}(\set{y_1,y_2})
  }\nonumber
\end{align}
\end{definition}



\subsection{Partitioning identity}

The members of the partition induced by $\equiv_{\indp_{\approx}}$
  are sets of resource pairs that share the same sharing properties.

Notice that the partitioned pairs contain but are not limited to
  the identity pairs.
Therefore, for sets of pairs closed under simi-discernibility
  we have the following three possibilities:
  \begin{enumerate}
    \item All pairs are identity pairs.
          This characterizes a consistent portion of the identity relation,
          since no simi-discernible pair is left out of this set.
    \item Some pairs are identity pairs.
          This characterizes a portion of the identity relation which is not
          applied consistently with respect to
          the simi-discernibility relation.
    \item No pairs are identity pairs.
          This characterizes a portion of the collection of pairs
          that is consistently kept out of the identity relation.
  \end{enumerate}

\noindent Each member of the simi-discernibility partition that is not
  of the third kind, i.e. every set of pairs that contains some identity pair,
  can be though of as an identity subrelation.
The simi-discernibility partition also partitions the identity relation
  into \emph{identity subrelations}.
Each identity subrelation can be described in terms of
  its discernibility predicates,
  i.e. in meaningful terms drawn from the domain vocabulary.

