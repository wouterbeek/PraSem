\documentclass[letterpaper]{article}

% Parts of the style depend on whether a PDF or a DVI output is created.
\usepackage{ifpdf}

\usepackage{aaai}
\usepackage{amsfonts}
\usepackage{amsmath}
\usepackage{amsthm}
\usepackage{courier}
\usepackage[english]{babel}

% Include graphics.
\usepackage{graphicx}
\ifpdf
  % Declare the supported file extensions.
  \DeclareGraphicsExtensions{.jpg,.mps,.pdf,.png}
\fi

\usepackage{helvet}
\usepackage[utf8]{inputenc}
\usepackage{times}
\usepackage{verbatim}
\usepackage{changebar}
\usepackage[normalem]{ulem}
\usepackage{url}
\newcommand{\URL}[1]{{\small \url{#1}}}

% Operator macros.
\newcommand{\absolute}[1]{\lvert#1\rvert}
\newcommand{\bigsetdef}[2]{\big\{#1\,\,\big\vert\,\,#2\big\}}
\newcommand{\card}[1]{\lvert#1\rvert}
\newcommand{\equivpair}[2]{#1 \approx #2}
\newcommand{\equivset}[1]{[#1]_{\approx}}
\newcommand{\higherapprox}[0]{\overline{\approx}}
\newcommand{\interp}[1]{#1^{\mathcal{I}}}
\newcommand{\lowerapprox}[0]{\underline{\approx}}
\newcommand{\natnum}[1]{#1 \in \mathbb{N}}
\newcommand{\pair}[2]{\langle#1,#2\rangle}
\newcommand{\powerset}[1]{\mathcal{P}(#1)}
\newcommand{\range}[2]{#1,\ldots,#2}
\newcommand{\set}[1]{\{#1\}}
\newcommand{\setdef}[2]{\{#1\,\vert\,#2\}}
\newcommand{\setrange}[2]{\{#1,\ldots,#2\}}
\newcommand{\triple}[3]{\langle#1,#2,#3\rangle}
\newcommand{\tuple}[1]{\langle#1\rangle}
\newcommand{\tuplerange}[2]{\langle#1,\ldots,#2\rangle}

% Operator declarations
\DeclareMathOperator{\indpo}{{\mathbb{IND}-\mathbb{PO}}}
\DeclareMathOperator{\indp}{{\mathbb{IND}-\mathbb{P}}}

% Theorem styles.
\newtheorem{assumption}{Assumption}
\newtheorem{axiom}{Axiom}
\newtheorem{convention}{Convention}
\newtheorem{example}{Example}
\newtheorem{definition}{Definition}
\newtheorem{lemma}{Lemma}
\newtheorem{principle}{Principle}
\newtheorem{proposition}{Proposition}
\newtheorem{specification}{Specification}
\newtheorem{statement}{Statement}
\newtheorem{theorem}{Theorem}
%\theoremstyle{definition}

\frenchspacing

\setlength{\pdfpagewidth}{8.5in}
\setlength{\pdfpageheight}{11in}

\pdfinfo{
/Title Rough Set Semantics for Identity on the Web
/Author Wouter Beek, Stefan Schlobach, Frank van Harmelen}
\setcounter{secnumdepth}{1}

\author{
  Wouter Beek \and Stefan Schlobach \and Frank van Harmelen\\
  Vrije Universiteit Amsterdam\\
  De Boelelaan 1081a\\
  1081HV Amsterdam\\
  The Netherlands
}

\title{Rough Set Semantics for Identity on the Web}

\begin{document}

\maketitle
\begin{abstract}
Identity relations are at the foundation of many
  logic-based knowledge representations.
We argue that the traditional notion of equality,
  is unsuited for many realistic knowledge representation settings.
The classical interpretation of equality is too strong
  when the equality statements are re-used outside their original context.
On the Semantic Web, equality statements are used to interlink
  multiple descriptions of the same object,
  using {\small \texttt{owl:sameAs}} assertions.
And indeed, many practical uses of {\small \texttt{owl:sameAs}}
  are known to violate the formal Leibniz-style semantics.

We provide a more flexible semantics to identity by assigning meaning to
  the subrelations of an identity relation
  in terms of the predicates that are used in a knowledge-base.
Using those indiscernability-predicates,
  we define upper and lower approximations of equality in the style of
  rought-set theory, resulting in a quality-measure for
  identity relations.
\end{abstract}

\section{Introduction}
\label{sec:intro}
\label{sec:relwork}

Clustering resources into different kinds can be useful
 for various purposes, e.g. browsing based on categories,
 providing automatic suggestions for ontology design.

We argue that RDF data deals with resources of different ``kinds'':
 two resources of the same kind will have similar relations
 to other resources.
We exploit this to partition the data into ``kinds'' automatically.

There are various ways to partition a set of resources into kinds.
The most straightforward approach is to use an explicit vocabulary
 in order to indicate
   that resources are members of the same set (extensional definition)
 or
   that resources are instance of the same class (intensional definition).
The vocabulary of RDF(S) \cite{BrickleyGuha2014} uses
 the intensional definition,
 denoted by the \texttt{rdf:type} property term.

In addition to used the explicit schema information,
 various methods have been explored that
 extract implicit schema information.
In \cite{NeumannMoerkotte2011},
 kinds of resources are identified
 by their \emph{characteristic set},
 which is the set properties that are asserted of a resource.

A comparison of the type-based and the property-based
 identification of resource kinds was performed by \cite{GottronKSS13},
 who showed that there is considerable mutual information between
 these two approaches.

A third approach is to use the object terms in addition to the predicate terms
 in order to characterize a resource.
In \cite{buikstra2011ranking}, the similarity between two resources
 is measured by the number of predicate-object pairs they have in common.

As we will argue in Section \ref{sec:fingerprints},
 the existing approaches towards partitioning a set of resources
 have shortcomings: they either identify too few partition members
 given the data at hand, or they identify too many partition members
 than can be substantiated by the data. 

In Section \ref{sec:approach} we give a generalization of
 these existing approaches and show that for arbitrary datasets
 more granular partitions can be found that are still validated
 by the data, i.e. that do not overfit.
Section \ref{sec:implementation} gives some implementation details,
 and Section \ref{sec:evaluation} shows the evaluation results we
 obtained by executing our implementation on existing datasets.
Section \ref{sec:conclusion} concludes.


\section{Related work}
\label{sec:related_work}

Existing research suggests six different solutions for
  the problem of identity on the SW.

\textbf{[1] Introduce weaker versions of {\small \texttt{owl:sameAs}}}
  \cite{HalpinHayes2010,MccuskerMcguinness2010}.
Candidates for replacement are
  the SKOS concepts
  {\small \texttt{skos:related}} and {\small \texttt{skos:exactMatch}}
  \cite{MilesBechhofer2009}.
The former is not transitive,
  thereby limiting the possibilities for reasoning.
The latter is transitive,
  but can only be used in certain contexts.
It is not defined in what contexts it can be used
  \cite{MilesBechhofer2009}.\footnote{
    For instance, the property {\small \texttt{skos:exactMatch}}
    ``is used to link two concepts, indicating a high degree of confidence
    that the concepts can be used interchangeably across a wide range of
    information retrieval applications.''
  }
\begin{comment}
% SIMILARITY
The problem with using weaker notions such as relatedness,
  is that everything is related to everything in \emph{some} way.}
% Shall we discuss similarity here as well?
% Does similarity differ from relatedness?
\end{comment}

\textbf{[2] Restrict the applicability of identity relations}
  to specific contexts.
In terms of Semantic Web technology, identities are expected to hold
  within a named graph or within a namespace,
  but not necessarily outside of it \cite{HalpinHayes2010}.
\cite{Melo2013} has successfully used the Unique Names Assumption
  within namespaces in order to identify many (arguably) spurious
  identity statements.

\textbf{[3] Introduce additional vocabulary} that does not weaken but extends
  the existing identity relation.
\cite{HalpinHayes2010} mention an explicit distinction that could be made
  between mentioning a term and using a term,
  thereby distinguishing an object and a Web document describing that object.
Other possible extensions of {\small \texttt{owl:sameAs}} might take
  the Fuzzyness and/or uncertainty of identity statements into account.

\textbf{[4] Use domain-specific identity relations}
  \cite{MccuskerMcguinness2010}.
For instance
    ``$x$ and $y$ have the same medical use''
  replaces
    identity in the domain of medicine,
and
    ``$x$ and $y$ are the same molecule''
  replaces
    identity in the domain of chemistry.
The downside to this solution is that domain-specific links are
  only locally valid, thereby limiting knowledge reuse.

\textbf{[5] Change the modeling practice}, possibly in a (semi-)automated way
  by adapting visualization and modeling toolkits to produce notifications
  upon reading SW data, or by posing additional restrictions on the creation
  and alteration of data. For example, adding an RDF link could require
  reciprocal confirmation from the maintainers of the respective datasets.
  \cite{HalpinHayes2010,DingShinavierFininMcguinness2010}
The problem with introducing checks on editing operations,
  is that it violates one of the fundamental underpinnings of the SW;
  namely that on the Web of Data anybody is allowed to say
  anything about anything \cite{AntoniouGrothHarmelenHoekstra2012}.

\textbf{[6] Extract network properties of {\small \texttt{owl:sameAs}}
  datasets} \cite{DingShinavierShangguanMcguinness2010}.
Although this work shows that network analysis can provide insights
  into the ways in which identity is used in the SW,
  these endeavors have not yet been related to the semantics of the
  identity relation.
We believe that utilizing network theoretic aspects in order to
  determine the meaning of identity statements
  would be interesting future research.

What the existing approaches have in common is
  that quite some work has to be done
  (adapting or creating standards, instructing modelers, converting existing
  datasets) in order to resolve only some of the problems of identity.
Our approach provides a way of dealing with the heterogeneous real-world
  usage of identity in the SW that is fully automated and requires
  no changes to standards, modeling practices, or existing datasets.

\section{Approach}
\label{sec:approach}

In contrast to the explicit schema information that should be present
in an RDF data set in the form of \texttt{rdf:type} specifications, in
this section we will use the structural properties of the graph to
extract implicit schema information.
structural properties of the graph can be used

In the previous section we saw that both excluding and including object terms
 in the partitioning of resources has negative effects on some data.
In this section we seek the best of both worlds
 and generalize these approaches using a partition $\M$ of resources.

Let $\tn{class}(t) = C$ iff $t \in C$ and $C \in \M$,
 thus the class of a term is its partition cell.
We now define the generic fingerprint (called $f_\M$) of a subject term $s$
 in Definition \ref{def:fingerprint}.

\begin{definition}[Fingerprint]
  \label{def:fingerprint}
  $
    f_\M(s)
  := \\
    \{
      \tuple{p,C} \in P_G \times \mathcal{P}(O_G)
    \mid
      \pair{I(s)}{I(o)} \in Ext(I(p)) \, \wedge \, o \in C \\
      \wedge \, C \in \M
    \}
  $
\end{definition}

The fingerprints $f_2$ and $f_3$ in Section \ref{sec:fingerprints}
 are instantiations of the generic fingerprint $f_\M$.
They correspond to, respectively,
  a partition $\M$ that contains a single set of all terms,
 and
  a partition $\M$ that contains all terms as singleton sets.

The advantage of generalizing these methods using $f_m$ is that
 a well chosen $\M$ can interpolate between these two existing approaches,
 and retain all relevant information about the term while discarding
 irrelevant details.
Our method of selecting a suitable partition is
 described in Section~\ref{sec:mdl} below.

\subsection{MDL Model Selection}
\label{sec:mdl}

A well motivated approach to model selection is based on
 the Minimum Description Length (MDL) principle
 \cite{Rissanen78,Rissanen84,grunwald2007}.
MDL can be interpreted as a formalization of Occam's Razor,
 which states that a more simple hypothesis,
 or model, should be preferred over a more complicated explanation of
 the same data.
In MDL, the words ``simple'' and ``complicated'' are
 made precise by using codes.
Given a space $\M$ of models, and data $D\in\D$,
 we have to specify:

\begin{itemize}
\item A code $C:\M\to\{0,1\}^*$ to encode the model
\item For each model $M\in\M$, a code $C_M:\D\to\{0,1\}^*$ to encode
  the data, making use of the model.
\end{itemize}

The second code uses the model in order to achieve an efficient encoding.
The best model is then the one that minimizes the overall code-length
 (equation \ref{eq:mdl}).

\begin{equation}
  \label{eq:mdl}
  M_\tn{mdl}:=\arg\min_{M\in\M}|C(M)| + |C_M(D)|.
\end{equation}

By balancing these two contributions
 (the coder and the data encoded using that codes)
 to the code length,
 MDL avoids very simple models that provide poor fit to the data
 (as these would have a long $C_M(D)$),
 and it also avoids overfitting,
 as overly complex models have a long $C(M)$.
Also note that the model selection procedure only relies on
 the code \emph{lengths}, so it suffices to define code length functions
 without worrying about the actual code words.
In the following we will write $L(x)$ as a shorthand for $|C(x)|$.

MDL model selection is very closely related to Bayes factors model selection,
 where the prior distribution on models replaces $C(M)$
and the likelihood function replaces $C_M(D)$;
readers more familiar with Bayesian methodology may wish to
 mentally make these substitutions and reading
 ``negative loglikelihood'' wherever we write ``code length''.

In our application,
 the model $\M$ is the partition that determines the fingerprints.
The construction of the code is outlined in the next section.




\section{Conclusion}
\label{sec:conclusion}

In this paper we presented a new approach for characterizing,
  extending, retracting, and assessing identity relations.
Our approach does this in purely qualitative terms, using schema semantics.

In section \ref{sec:introduction} we enumerated three research goals.
The first goal is met, since an indiscernibility partition characterizes
  identity subrelations based on the predicates $P$ (closed under identity)
  for which the pairs in that sets are indiscernible.
In this way we can distinguish between different types of identity
  by treating $P$ as a description of a (sub)set of identity pairs.
We suggest that the meaning of an identity relation and its subrelations
  is partially defined in its use,
  i.e., in the indiscernibility criteria it embodies.

The second goal is met, since the notion of a rough set allows us to
  distinguish between pairs that must be (lower approximation)
  and those that may be (higher approximation)
  % 'may' = 'not must not'
  in the identity relation.
If we want to add/remove pairs of the identity relation,
  we should not consider pairs of the former but only pairs of
  the latter kind.

The third goal is met, since the measure for rough set accuracy
  is based on the discernibility criteria of an identity set.
The crispness of the set is proportional to the quality of the
  identity relation, based on its semantic consistency.



% Bibliography
\bibliographystyle{aaai}
\bibliography{iotw,prasem,web_standards}

\end{document}

